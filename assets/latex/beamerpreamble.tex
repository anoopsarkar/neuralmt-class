\documentclass[xcolor=table,dvipsnames,table]{beamer}
\mode<presentation>
\usetheme{boxes}
\setbeamertemplate{navigation symbols}{}
% http://www.latex-community.org/forum/viewtopic.php?f=4&t=6694
\setbeamertemplate{navigation symbols}{\raisebox{5pt}{\makebox[\paperwidth]{\hfill\makebox[10pt]{\scriptsize\insertframenumber\vspace{1ex}}}}}
%\setbeamertemplate{footline}[frame number]
\setbeamertemplate{blocks}[shadow=false]
%\setbeamercolor*{block title}{fg=structure,bg=RoyalBlue!10}
\setbeamercolor*{block title example}{fg=structure,bg=RoyalBlue!10}
%\setbeamercolor*{block title example}{fg=BrickRed,bg=Goldenrod!10}
\setbeamercolor*{block title alerted}{fg=white,bg=black}
\addtobeamertemplate{block begin}{\pgfsetfillopacity{0.8}}{\pgfsetfillopacity{1}}
%\rowcolors{0}{RoyalBlue!20}{RoyalBlue!5}

%\DeclareGraphicsRule{*}{mps}{*}{}

\usepackage{latexsym}
\usepackage{hyperref}
\usepackage{tikz}
\usetikzlibrary{calc,shapes,arrows,shadows,shapes.callouts,shapes.arrows,chains,positioning,trees}
\usepackage{solution}
\usepackage{calc}
\usepackage{pifont}
\usepackage{algorithmic}

\newcommand{\cmark}{\ding{51}}
\newcommand{\xmark}{\ding{55}}

\newcounter{mycallout}

\newcommand{\callouts}[3]{%
  \stepcounter{mycallout}
  \tikz[remember picture,baseline]{\node[anchor=base,inner sep=0,outer sep=0]%
    (\themycallout) {\colorbox{#1!20}{#3}};\pause\node[overlay,rectangle callout,%
    callout relative pointer={(0cm,0.5cm)},fill=#1!20] at ($(\themycallout.south)+(-0cm,-0.7cm)$){#2};}%
    }%

\raggedright

\newcount\lecturecount
\lecturecount=0
\AtBeginLecture{%
    \advance\lecturecount by 1
    \date{}
    \begin{frame}
    \begin{center}
    \titlepage
    \ifnum\lecturecount=1
    Part \the\lecturecount: \insertlecture
    \else
    Part \the\lecturecount: \insertlecture
    \fi
    \end{center}
    \end{frame}
}

\addtobeamertemplate{block begin}{\setlength\abovedisplayskip{0pt}}

%\newcommand{\example}[1]{{\color{BrickRed!50}{#1}}}
\newcommand{\maths}[1]{{\color{RoyalBlue!50}{#1}}}
\newcommand{\reference}[1]{{\color{RoyalBlue!30}\tiny [from #1]}}
\newcommand{\koehnref}{\reference{\href{http://www.statmt.org/book}{P.Koehn SMT book slides}}}

